\chapter{Podsumowanie}
\label{chap:summary}

Niewątpliwy postęp w dziedzinie uczenia maszynowego i głębokich sieci konwolucyjnych doprowadził do przełomu na wielu płaszczyznach. Najnowsze doniesienia z branży, na które zabrakło niestety miejsca w tej pracy dotyczą przykładowo automatycznego generowania, nigdy wcześniej nie istniejących, realistycznych ludzkich twarzy.
Dalszy rozwój tej dziedziny niewątpliwie doprowadzi do automatyzacji procesów, w których obecnie udział człowieka może wydawać się niezbędny. Jesteśmy dopiero na początku tego procesu co sprawia, że możemy spodziewać się kolejnych przełomowych osiągnięć w kolejnych latach.

Postęp w rozwoju sieci konwolucyjnych jest możliwy dzięki ich lepszemu zrozumieniu. Wizualizacje działania sieci neuronowych ułatwiają przyswajanie wiedzy na ich temat oraz często prowadzą do wynajdywania nowych dla nich zastosowań. Widzą to zarówno pracujący na uczelniach naukowcy i szerogowi pracownicy firm IT. Wspaniale jest widzieć, jak przy współpracy uczelnii i firm powstają takie narzędzia jak \textit{lucid}. Świetne jest to, że uczelnie udostępniają wytrenowane
wagi sieci neuronowych tak by każdy mógł dokonywać na nich swoich eksperymentów.

Oprócz twórczego aspektu, takie eksperymentowanie naprowadza nas na kolejne, lepsze architektury konwolucyjnych sieci neurnowych. O ile w przypadku analizy języka naturalnego, już dziś posiadamy odpowiednią ilość danych w stosunku do wydajności tam stosowanych architektur, tak w przypadku analizy obrazu przy użyciu sieci neuronowych wciąż jest ogromne pole do zagospodarowania dla innowacji.

Uważam, że uzyskane wizualizacje obok bezpośrednich efektów dydaktycznych dla mnie, były świetną okazją by zaznajomić się z dalszą teorią stojącą za sieciami neuronowymi i mam nadzieję na dalszy swój rozwój w tej dziedzinie.
