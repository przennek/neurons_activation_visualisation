\chapter{Podsumowanie}
\label{chap:summary}

Niewątpliwy postęp w dziedzinie uczenia maszynowego i głębokich sieci konwolucyjnych doprowadził do przełomu na wielu płaszczyznach. Najnowsze doniesienia z branży, na które zabrakło niestety miejsca w tej pracy dotyczą przykładowo automatycznego generowania, nigdy wcześniej nie istniejących, realistycznych ludzkich twarzy.
Jest to niezwykle fascyjnujące, choć bywa też niebezpieczne, biorąc pod uwagę, że jesteśmy w stanie przy pomocy \textit{CNN} podmienić jedną twarz na drugą w czasie rzeczywistym - przy pomocy narzędzia nazwanego \textit{DeepFake}. W chwili gdy spisuje te słowa minęły dwa dni od doniesienia Niebezpiecznik.pl na temat pierwszego udanego wyłudzenia przy pomocy spreparowanego przy rekurencyjną siecią neuronowej głosu. 
Ta technologia w połączeniu z \textit{DeepFake} otwiera zupełnie nowe możliwości dla wszelkiej maści oszustów ale też popularnej w mediach manipulacji.

Pozostaje nam się mieć na baczności i czerpać korzyści z pozytywnych aspektów \textit{CNN}. Postęp w ich rozwoju, możliwy dzięki ich lepszemu zrozumieniu, osiągnięty, między innymi, wizualizacjami sprawia, że wiele czynności wykonywanych przez ludzi może zostać zautomatyzowana.
Pomijając problematyczny aspekt tego w jaki sposób ludzi przekwalifikować, to zmniejszone koszty leczenia ludzi przy pomocy automatyzacji pracy radiologa są niewątpliwie pozytywnym zjawiskiem w skali globalnej.

By to wszystko było możliwe ilość danych, którymi dysponujemy musi wielokrotnie wzrosnąć. 
