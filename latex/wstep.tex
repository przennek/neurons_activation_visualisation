\chapter{Wstęp}
\label{cha:wstep}

Rozwój nauki oraz stały wzrost mocy obliczeniowej komputerów pozwolił na rzeczy, które wydawałby się niemożliwe jeszcze dekadę temu. Chociaż pierwsze ważne dla sieci neuronowych koncepty sięgają lat pięćdziesiątych, a kluczowe kwestie takie jak algorytm propagacji wstecznej lat siedemdziesiątych, to dopiero w ciągu ostatnich dziesięciu lat nastąpiła popularyzacja tej technologii na nie spotykaną dotąd skalę.

Przy pomocy sieci neuronowych zaczęto budować systemy przewidujące ceny nieruchomości, indeksy giełdowe czy to, jakie jest prawdopodobieństwo tego, że ktoś zapadnie na przewlekłą chorobę. Co więcej, wiele firm zbudowało na tych modelach ogromny kapitał. Przykładowo, sieci posłużyły im do predykcji zainteresowania danym produktem. W 2014 roku Internet obiegła informacja, że firma Amazon przewiduje to, co ich klient zamierza kupić zanim jeszcze złoży zamówienie. Między innymi, dzięki temu udało im się skrócić czas oczekiwania na zamówienie do 48 godzin.

Ale na przewidywaniach się nie skończyło. Rekurencyjne sieci neuronowe rozpoznają słowa kluczowe i sprawiają, że możemy wydawać polecenia naszym urządzeniom. Co więcej, z ich pomocą tłumaczenia prosto z internetowego translatora brzmią z dnia na dzień coraz bardziej naturalnie. Automatyczne transkrypcje są coraz szerzej dostępne pod filmami udostępnionymi w Internecie a ich poprawność rośnie z dnia na dzień.

Komputery zaczęły rozpoznawać przedmioty zapisane na wideo i zdjęciach.  Ta futurystyczna technologia wspomaga rzeczy tak prozaiczne jak kontrola jakości czipsów na liniach produkcyjnych. Automatyczne sortownie odpadów mieszanych przynoszą nieoceniony zysk dla środowiska i realny, liczony w dolarach zysk dla ich właścicieli. 

W mojej pracy skupię się na sieciach z dziedziny przetwarzania obrazu. Po wprowadzeniu w podstawowe wiadomości z dziedziny głębokich sieci neuronowych, zaprezentuję budowę prostego modelu takiej sieci, na przykładzie neuronowej sieci konwolucyjnej (ang. \textit{Convolutional Neural Networks} – \textit{CNN}) LeNet-5. 
Przy jej pomocy, wpierw sklasyfikuję kilka prostych symboli, a następnie, zwizualizuję, czego tak naprawdę nauczył się mój automat.

Przybliżę również architekturę sieci VGG oraz bardziej złożone wizualizacje na podstawie jej warstw. Uzyskane zostaną one poprzez maksymalizację mediany wygenerowanych aktywacji dla obrazu wyjściowego.

Opiszę zastosowanie sieci VGG w generowaniu artstycznych obrazów przy pomocy neuronowego transferu stylu (ang. \textit{Neural Style Transfer}). 

Pod koniec mojej pracy, wprowadzę koncepcję sieci typu incepcja (ang. \textit{Inception network}) i na jej
podstawie wytłumaczę na czym polega \textit{DeepDream} zaprezentowany w 2015 roku przez inżynierów z Google.

Mam nadzieję, że każdy po zapoznaniu się z treścią tej pracy, będzie lepiej zaznajomiony z tematem głębokich sieci neuronowych służących do przetwarzania obrazu.
