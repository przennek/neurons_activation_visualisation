\chapter{Wstęp}
\label{cha:wstep}

Rozwój nauki oraz stały wzrost mocy obliczeniowej komputerów pozwolił na rzeczy, które wydawałby się niemożliwe jeszcze dekadę temu. Chociaż pierwsze ważne dla sieci neuronowych koncepty sięgają lat pięćdziesiątych, a kluczowe kwestie takie jak algorytm propagacji wstecznej lat siedemdziesiątych, to dopiero w ciągu ostatnich dziesięciu lat nastąpiła popularyzacja tej technologii na nie spotykaną dotąd skalę.

Przy pomocy sieci neuronowych zaczęto budować systemy przewidujące ceny nieruchomości, indeksy giełdowe czy to jakie jest prawdopodobieństwo na to, że ktoś zapadnie na przewlekłą chorobę. Co więcej, wiele firm zbudowało na tych modelach ogromny kapitał. Przykładowo, sieci posłużyły im do predykcji zainteresowania danym produktem. W 2014 roku Internet obiegła informacja, że firma Amazon przewiduje to co ich klient zamierza kupić zanim jeszcze złoży zamówienie. Między innymi, dzięki temu udało im się skrócić czas oczekiwania na zamówienie do 48 godzin.

Ale na przewidywaniach się nie skończyło. Rekurencyjne sieci neuronowe rozpoznają słowa kluczowe i sprawiają, że możemy wydawać polecenia naszym urządzeniom. Co więcej, z ich pomocą tłumaczenia prosto z internetowego translatora brzmią z dnia na dzień coraz bardziej naturalnie. Automatyczne transkrypcje są coraz szerzej dostępne pod filmami udostępnionymi w Internecie a ich poprawność rośnie z dnia na dzień.

Komputery zaczęły rozpoznawać przedmioty zapisane na wideo i zdjęciach.  Ta futurystyczna technologia wspomaga rzeczy tak prozaiczne jak kontrola jakości czipsów na liniach produkcyjnych. Automatyczne sortownie odpadów mieszanych przynoszą nieoceniony zysk dla środowiska i realny, liczony w dolarach zysk dla ich właścicieli. 

W mojej pracy skupię się na sieciach z dziedziny przetwarzania obrazu. Przy pomocy neuronowych sieci konwolucyjnych (z angielskiego Convolutional Neural Networks – CNN) wpierw sklasyfikuję kilka prostych symboli przy pomocy sieci konwolucyjnej LeNet-5, a następnie zwizualizuję, czego tak naprawdę nauczył się mój automat.

Przybliżę również architekturę sieci VGG-16 a wraz z nią technikę NSS (z angielskiego Neural Style Transfer). Przy jej pomocy zwizualizuję cechy nauczone przez model.
